% Don't like 10pt? Try 11pt or 12pt
\documentclass[10pt]{article}

% This is a helpful package that puts math inside length specifications
\usepackage{calc}


% Simpler bibsection for CV sections
% (thanks to natbib for inspiration)
\makeatletter
\newlength{\bibhang}
\setlength{\bibhang}{1em}
\newlength{\bibsep}
 {\@listi \global\bibsep\itemsep \global\advance\bibsep by\parsep}
\newenvironment{bibsection}%
        {\vspace{-\baselineskip}\begin{list}{}{%
       \setlength{\leftmargin}{\bibhang}%
       \setlength{\itemindent}{-\leftmargin}%
       \setlength{\itemsep}{\bibsep}%
       \setlength{\parsep}{\z@}%
        \setlength{\partopsep}{0pt}%
        \setlength{\topsep}{0pt}}}
        {\end{list}\vspace{-.6\baselineskip}}
\makeatother

% Layout: Puts the section titles on left side of page
\reversemarginpar

% Add Japanese!

%
%         PAPER SIZE, PAGE NUMBER, AND DOCUMENT LAYOUT NOTES:
%
% The next \usepackage line changes the layout for CV style section
% headings as marginal notes. It also sets up the paper size as either
% letter or A4. By default, letter was used. If A4 paper is desired,
% comment out the letterpaper lines and uncomment the a4paper lines.
%
% As you can see, the margin widths and section title widths can be
% easily adjusted.
%
% ALSO: Notice that the includefoot option can be commented OUT in order
% to put the PAGE NUMBER *IN* the bottom margin. This will make the
% effective text area larger.
%
% IF YOU WISH TO REMOVE THE ``of LASTPAGE'' next to each page number,
% see the note about the +LP and -LP lines below. Comment out the +LP
% and uncomment the -LP.
%
% IF YOU WISH TO REMOVE PAGE NUMBERS, be sure that the includefoot line
% is uncommented and ALSO uncomment the \pagestyle{empty} a few lines
% below.
%

%% Use these lines for letter-sized paper
\usepackage[paper=letterpaper,
            %includefoot, % Uncomment to put page number above margin
            marginparwidth=1.2in,     % Length of section titles
            marginparsep=.05in,       % Space between titles and text
            margin=1in,               % 1 inch margins
            includemp]{geometry}

%% Use these lines for A4-sized paper
%\usepackage[paper=a4paper,
%            %includefoot, % Uncomment to put page number above margin
%            marginparwidth=30.5mm,    % Length of section titles
%            marginparsep=1.5mm,       % Space between titles and text
%            margin=25mm,              % 25mm margins
%            includemp]{geometry}

%% More layout: Get rid of indenting throughout entire document
\setlength{\parindent}{0in}

%% This gives us fun enumeration environments. compactitem will be nice.
\usepackage{paralist}

%% Reference the last page in the page number
%
% NOTE: comment the +LP line and uncomment the -LP line to have page
%       numbers without the ``of ##'' last page reference)
%
% NOTE: uncomment the \pagestyle{empty} line to get rid of all page
%       numbers (make sure includefoot is commented out above)
%
\usepackage{fancyhdr,lastpage}
\pagestyle{fancy}
%\pagestyle{empty}      % Uncomment this to get rid of page numbers
\fancyhf{}\renewcommand{\headrulewidth}{0pt}
\fancyfootoffset{\marginparsep+\marginparwidth}
\newlength{\footpageshift}
\setlength{\footpageshift}
          {0.5\textwidth+0.5\marginparsep+0.5\marginparwidth-2in}
\lfoot{\hspace{\footpageshift}%
       \parbox{4in}{\, \hfill %
                    \arabic{page} of \protect\pageref*{LastPage} % +LP
%                    \arabic{page}                               % -LP
                    \hfill \,}}

% Finally, give us PDF bookmarks
\usepackage{color,hyperref}
\definecolor{darkblue}{rgb}{0.0,0.0,0.3}
\hypersetup{colorlinks,breaklinks,
            linkcolor=darkblue,urlcolor=darkblue,
            anchorcolor=darkblue,citecolor=darkblue}

%%%%%%%%%%%%%%%%%%%%%%%% End Document Setup %%%%%%%%%%%%%%%%%%%%%%%%%%%%


%%%%%%%%%%%%%%%%%%%%%%%%%%% Helper Commands %%%%%%%%%%%%%%%%%%%%%%%%%%%%

% The title (name) with a horizontal rule under it
% (optional argument typesets an object right-justified across from name
%  as well)
%
% Usage: \makeheading{name}
%        OR
%        \makeheading[right_object]{name}
%
% Place at top of document. It should be the first thing.
% If ``right_object'' is provided in the square-braced optional
% argument, it will be right justified on the same line as ``name'' at
% the top of the CV. For example:
%
%       \makeheading[\emph{Curriculum vitae}]{Your Name}
%
% will put an emphasized ``Curriculum vitae'' at the top of the document
% as a title. Likewise, a picture could be included:
%
%   \makeheading[\includegraphics[height=1.5in]{my_picutre}]{Your Name}
%
% the picture will be flush right across from the name.
\newcommand{\makeheading}[2][]%
        {\hspace*{-\marginparsep minus \marginparwidth}%
         \begin{minipage}[t]{\textwidth+\marginparwidth+\marginparsep}%
             {\large \bfseries #2 \hfill #1}\\[-0.15\baselineskip]%
                 \rule{\columnwidth}{1pt}%
         \end{minipage}}

% The section headings
%
% Usage: \section{section name}
%
% Follow this section IMMEDIATELY with the first line of the section
% text. Do not put whitespace in between. That is, do this:
%
%       \section{My Information}
%       Here is my information.
%
% and NOT this:
%
%       \section{My Information}
%
%       Here is my information.
%
% Otherwise the top of the section header will not line up with the top
% of the section. Of course, using a single comment character (%) on
% empty lines allows for the function of the first example with the
% readability of the second example.
\renewcommand{\section}[2]%
        {\pagebreak[3]\vspace{1.3\baselineskip}%
         \phantomsection\addcontentsline{toc}{section}{#1}%
         \hspace{0in}%
         \marginpar{
         \raggedright \scshape #1}#2}

% An itemize-style list with lots of space between items
\newenvironment{outerlist}[1][\enskip\textbullet]%
        {\begin{itemize}[#1]}{\end{itemize}%
         \vspace{-.6\baselineskip}}

% An environment IDENTICAL to outerlist that has better pre-list spacing
% when used as the first thing in a \section
\newenvironment{lonelist}[1][\enskip\textbullet]%
        {\vspace{-\baselineskip}\begin{list}{#1}{%
        \setlength{\partopsep}{0pt}%
        \setlength{\topsep}{0pt}}}
        {\end{list}\vspace{-.6\baselineskip}}

% An itemize-style list with little space between items
\newenvironment{innerlist}[1][\enskip\textbullet]%
        {\begin{compactitem}[#1]}{\end{compactitem}}

% An environment IDENTICAL to innerlist that has better pre-list spacing
% when used as the first thing in a \section
\newenvironment{loneinnerlist}[1][\enskip\textbullet]%
        {\vspace{-\baselineskip}\begin{compactitem}[#1]}
        {\end{compactitem}\vspace{-.6\baselineskip}}

% To add some paragraph space between lines.
% This also tells LaTeX to preferably break a page on one of these gaps
% if there is a needed pagebreak nearby.
\newcommand{\blankline}{\quad\pagebreak[3]}
\newcommand{\halfblankline}{\quad\vspace{-0.5\baselineskip}\pagebreak[3]}

% Uses hyperref to link DOI
\newcommand\doilink[1]{\href{http://dx.doi.org/#1}{#1}}
\newcommand\doi[1]{doi:\doilink{#1}}

% For \url{SOME_URL}, links SOME_URL to the url SOME_URL
\providecommand*\url[1]{\href{#1}{#1}}
% Same as above, but pretty-prints SOME_URL in teletype fixed-width font
\renewcommand*\url[1]{\href{#1}{\texttt{#1}}}

% For \email{ADDRESS}, links ADDRESS to the url mailto:ADDRESS
\providecommand*\email[1]{\href{mailto:#1}{#1}}
% Same as above, but pretty-prints ADDRESS in teletype fixed-width font
%\renewcommand*\email[1]{\href{mailto:#1}{\texttt{#1}}}

%\providecommand\BibTeX{{\rm B\kern-.05em{\sc i\kern-.025em b}\kern-.08em
%    T\kern-.1667em\lower.7ex\hbox{E}\kern-.125emX}}
%\providecommand\BibTeX{{\rm B\kern-.05em{\sc i\kern-.025em b}\kern-.08em
%    \TeX}}
\providecommand\BibTeX{{B\kern-.05em{\sc i\kern-.025em b}\kern-.08em
    \TeX}}
\providecommand\Matlab{\textsc{Matlab}}

%%%%%%%%%%%%%%%%%%%%%%%% End Helper Commands %%%%%%%%%%%%%%%%%%%%%%%%%%%

%%%%%%%%%%%%%%%%%%%%%%%%% Begin CV Document %%%%%%%%%%%%%%%%%%%%%%%%%%%%

\begin{document}



\makeheading{Emil Rehnberg}

%
% Contact Information
%
\section{Contact Information}
\newlength{\rcollength}\setlength{\rcollength}{2.15in}%1.85in
\begin{tabular}[t]{@{}p{\textwidth-\rcollength}p{\rcollength}}
Sk\"{o}ntorpsv\"{a}gen 102	& \textit{Mobile:} +46-722333135 \\
12053 \AA rsta	& \textit{E-mail:} \email{emil.rehnberg@gmail.com}\\
Sverige
\end{tabular}

%
% Citizinship
%
\section{Citizenship}
Swedish and Finnish

%
% Education
%
\section{Education}
\href{http://www.su.se/}{\textbf{Stockholm University}}, Stockholm, SWEDEN
\begin{outerlist}
\item[]	Magister Scientae, \href{http://math.su.se/}{Mathematical Statistics}, May 2009
        \begin{innerlist}
        \item Thesis Topic: \emph{Evaluation of a multipoint method for imputing genotypes using HapMap III}
        \item Supervisors:
              \href{juni.palmgren@ki.se}{Juni Palmgren}
              \href{keith.humphreys@ki.se}{Keith Humphreys}
              \href{monica.leu@ki.se}{Monica Leu}
        \item Area of Study: Biostatistics (genetics)
        \end{innerlist}
\end{outerlist}

%
% Professional experience
%
\section{Professional / Academic Experience}
%
\href{http://www.ncc.go.jp/en/nccri/divisions/14carc/14carc.html}{\textbf{Division of Epigenomics}}
\href{http://www.ncc.go.jp/}{\textbf{National Cancer Centre}}, Tsukiji, Tokyo, Japan
\begin{outerlist}

\item[] \textit{Biostatistician}%
	\hfill \textbf{February 2012 to May 2013}
\begin{innerlist}
\item Statistical consultant / informatician for researchers (Geneticists, Medical doctors).
\item Researcher
\item Bioinformatician
\end{innerlist}
\end{outerlist}

\halfblankline


\href{http://ki.se/ki/jsp/polopoly.jsp?d=9600}{\textbf{Department of Medical Epidemiology and Biostatistics}}
\href{http://www.ki.se/}{\textbf{Karolinska Institutet}}, Solna, Stockholm, Sweden
\begin{outerlist}

\item[] \textit{Biostatistician}%
        \hfill \textbf{December 2009 to January 2012}
\begin{innerlist}
\item Statistical consultant for multiple research projects.
\item Statistical consultant for PhD-students. Project planning, analysis, data management, methods.
\item Teaching assistant in biostatistic courses.
\end{innerlist}

\item[] \textit{Research Assistant}%
        \hfill \textbf{June 2009 to December 2009}
\\ Collaborators: Juni Palmgren, Keith Humphreys and Monica Leu
\begin{innerlist}
  \item Imputation of genotypes for NordicDB
  \begin{innerlist}
    \item Responsible for imputation of genotype data, genome-wide for NordicDB, which is a nordic database for genome-wide genetic information from controls (genotypes and allele frequences).
    \item Paper: NordicDB: a Nordic pool and portal for genome-wide control data, European Journal of Human Genetics , (28 July 2010)
  \end{innerlist}
\end{innerlist}

\end{outerlist}

\halfblankline

\href{http://www.matteakuten.se}{\textbf{Matteakuten}}, Stockholm, Sweden
\begin{outerlist}

\item[] \textit{Math teacher}%
        \hfill \textbf{September 2007 to January 2008}
\begin{innerlist}
\item Employed as a math teacher at upper secondary school level. (As extra work during university studies) 
\end{innerlist}

\end{outerlist}

%
% Submitted Journal Publications
%
\section{Submitted\\Journal\\Publications} \begin{bibsection}
\item Kim JG er al.
	Comprehensive DNA methylation and extensive mutation analyses reveal an association between the CpG island methylator phenotype and oncogenic mutations in gastric cancers.
	\emph{Cancer Letters} 2012 Nov 27
\item Scott RA et al.
	Large-scale association analyses identify new loci influencing glycemic traits and provide insight into the underlying biological pathways.
	\emph{Nat Genet.} 2012 Aug 12
    \item DIAGRAM consortium
	Large-scale association analysis provides insights into the genetic architecture and pathophysiology of type 2 diabetes.
	\emph{Nat Genet.} 2012 Aug 12
    \item Manning AK et al.
	A genome-wide approach accounting for body mass index identifies genetic variants influencing fasting glycemic traits and insulin resistance.
	\emph{Nat Genet.} 2012 May 13
    \item Stolk L et al.
	Meta-analyses identify 13 loci associated with age at menopause and highlight DNA repair and immune pathways.
	\emph{Nat Genet.} 2012 Jan 22
    \item Smedby KE et. al.
	GWAS of Follicular Lymphoma Reveals Allelic Heterogeneity at 6p21.32 and Suggests Shared Genetic Susceptibility with Diffuse Large B-cell Lymphoma.
	\emph{PLoS Genetics} 2011 April. 
    \item Monica Leu, Keith Humphreys, Ida Surakka, Emil Rehnberg, .. Juni Palmgren, Samuli Ripatti. 
	NordicDB: A Nordic pool and portal for genome-wide control data. 
	\emph{European Journal of Human Genetics}. 2010 December.
    \item Emil Rehnberg. 
	Evaluation of a multipoint method for imputing genotypes using HapMap III.
	Master’s thesis, Karolinska Institutet, Department of Medical Epidemiology and Biostatistics, 2009.
\end{bibsection}

%
% Papers in preparation 
%
%\section{Papers in Preparation} \begin{bibsection}
%    \item Kathy Lunetta et. al.
%	Meta-analysis of age at menopause identifies multiple novel loci highlighting DNA repair and aging pathways.
%    \item Linda Halldner, Emil Rehnberg, Niklas L\aa ngstr\"{o}m, Cecilia Lundholm, Marcus Boman, Paul Lichtenstein.
%        Birth month as a risk factor for ADHD in Sweden.
%\end{bibsection}

%
% Teaching experience
%
\section{Teaching Experience}
\href{http://ki.se/ki/jsp/polopoly.jsp?d=9600}{\textbf{Department of Medical Epidemiology and Biostatistics}}, Solna, Stockholm, Sweden
\begin{outerlist}
\item[] \textit{Teaching Assistant}
    \hfill \textbf{December 2009 to May 2011}
    \begin{innerlist}

        \item Master's Programme in Biomedicine, Biostatistics course
        \begin{innerlist}
	    \item Teaching assistant for pen \& paper exercises and computer labs. Descriptive, univariate, multivariate analysis, logistic regression. 
        \end{innerlist}

        \halfblankline

        \item Bachelor's Programme in Biomedicine, Biostatistics course
        \begin{innerlist}
	    \item R and R-commander introduction lecture.
	    \item Teaching assistant for pen \& paper exercises and computer labs.
        \end{innerlist}

        \halfblankline

	\item Biostatistics for Molecular Oncology course.
        \begin{innerlist}
            \item Statistical consulting for group discussions. Descriptive stats and survival analysis in R.
	    \item Teaching assistant for pen \& paper exercises and computer labs.
        \end{innerlist}

    \end{innerlist}
\end{outerlist}

\section{Professional activities}
\textit{The Swedish Society for Medical Statistics - 2011}
\begin{innerlist}
    \item{Board member}
    \item{Website manager}
\end{innerlist}

%
% Expertise
%
\section{Expertise}
\textit{Mathematics:}
\begin{innerlist}
    \item Applied mathematics, linear algebra, multivariate analysis
\end{innerlist}

\halfblankline

\textit{Statistics/Biostatistics:}
\begin{innerlist}
    \item Epidemiology, descriptive statistics, multivariate analysis, analysis of cathegorical data, stochastic processes, probability theory, survival analysis, prediction, principal component analysis, hierarchical clustering, supervised classification, motif analysis.
\end{innerlist}

%
% Software skills
%
\section{Software Skills}
\textit{Programming:}
\begin{innerlist}
    \item Advanced knowledge: R, SAS, Bash shell scripting.
    \item Basic knownledge: Ruby, Rails, parallel computing.
    \item Exposure to: SQL, HTML, Haml, Pascal, Scheme, Python, C\#, Fortran, C.
\end{innerlist}

\halfblankline

\textit{Applications:}
\begin{innerlist}
\item PLINK, multiple UNIX-based genetic software.
\end{innerlist}

\halfblankline

\textit{Productivity Applications:}
\begin{innerlist}
\item \TeX{}, Vim, git, UNIX tools such as: awk, sed, tmux, screen, etc.
\end{innerlist}

\halfblankline

\textit{Operating systems:}
\begin{innerlist}
    \item Ubuntu, Apple OS X and other UNIX variants.
\end{innerlist}


%
% Courses
%
\section{Courses}
\textit{Workshops / Lectures}
\begin{innerlist}
    \item RegStat 2011 - Extensions to Epidemiological Designs in Register-Based Research.
    \item ENGAGE IT-Course and Statistical Workshop: Tools for data analysis and management in complex traits genetic studies.
    \item Age Period-Cohort Modelling.
    \item Essentials of descriptive cancer epidemiology.
\end{innerlist}

\halfblankline

\textit{Courses}
\begin{innerlist}
    \item PDC Summer School - Introduction to High-Performance Computing.
    \item Epidemiological theory in a statistical framework.
\end{innerlist}


%
% Language skills
%
\section{Language Skills}
Swedish, English, Japanese (elementary)


\end{document}

%%%%%%%%%%%%%%%%%%%%%%%%%% End CV Document %%%%%%%%%%%%%%%%%%%%%%%%%%%%%

%----------------------------------------------------------------------%
% The following is copyright and licensing information for
% redistribution of this LaTeX source code; it also includes a liability
% statement. If this source code is not being redistributed to others,
% it may be omitted. It has no effect on the function of the above code.
%----------------------------------------------------------------------%
% Copyright (c) 2007, 2008, 2009, 2010, 2011 by Theodore P. Pavlic
%
% Unless otherwise expressly stated, this work is licensed under the
% Creative Commons Attribution-Noncommercial 3.0 United States License. To
% view a copy of this license, visit
% http://creativecommons.org/licenses/by-nc/3.0/us/ or send a letter to
% Creative Commons, 171 Second Street, Suite 300, San Francisco,
% California, 94105, USA.
%
% THE SOFTWARE IS PROVIDED "AS IS", WITHOUT WARRANTY OF ANY KIND, EXPRESS
% OR IMPLIED, INCLUDING BUT NOT LIMITED TO THE WARRANTIES OF
% MERCHANTABILITY, FITNESS FOR A PARTICULAR PURPOSE AND NONINFRINGEMENT.
% IN NO EVENT SHALL THE AUTHORS OR COPYRIGHT HOLDERS BE LIABLE FOR ANY
% CLAIM, DAMAGES OR OTHER LIABILITY, WHETHER IN AN ACTION OF CONTRACT,
% TORT OR OTHERWISE, ARISING FROM, OUT OF OR IN CONNECTION WITH THE
% SOFTWARE OR THE USE OR OTHER DEALINGS IN THE SOFTWARE.
%----------------------------------------------------------------------%
